% This file is part of staffAuth.
%
% staffAuth is a free document: you can redistribute it and/or modify it under
% the terms of the GNU General Public License as published by the Free Software
% Foundation, either version 3 of the License, or (at your option) any later
% version.
%
% This document is distributed in the hope that it will be useful,but WITHOUT
% ANY WARRANTY; without even the implied warranty of MERCHANTABILITY or
% FITNESS FOR A PARTICULAR PURPOSE. See the GNU General Public License for
% more details.
%
% You should have received a copy of the GNU General Public License along with
% this document. If not, see
%
%   http://www.gnu.org/licenses/
%
%
% Copyright (C)
%   2015-2016 Alexander Haase IT Services <support@alexhaase.de>
%


\section{Idea}

To solve these issues, a centralized database backend should be used, so every
employee is able to manage his own account with low overhead. Usually you'll
already have a central storage for your employees credentials to authenticate
them at your companys machines, so why don't use them? These data should be
accessed via a well defined API from outside, so you're free to use a backend of
your choice. \\

In general no passwords should be used at you customers' side. For login into
machines SSH-keys should be used, thus this is an asymmetric system. The
SSH-keys should be queried via API. For privileged access (sudo) onetime
passwords should be used. Thus algorithms like TOTP\footnote{Time-Based One-Time
Password Algorithm is defined in
\href{https://tools.ietf.org/html/rfc6238}{RFC 6238}} are symmetric, not the
client but the credential server should check the onetime password via API. \\

To avoid lots of configuration, a proxy server in the customers' network will be
used, which can be found via DNS. The server should be authenticated to avoid
security issues with DNS redirection attacks.


\subsection{Basic structure}

\begin{itemize}
	\item DNS will be used to find a local authentication proxy server in secure
		networks.
	\item The API will be accessed over HTTPS, so authentication can be
		ensured with TLS certificates.
	\item The proxy will pass requests to the API server to be used for the
		calling user. These proxy may authenticate himself with a TLS client
		certificate or API token to the API server, if required.
\end{itemize}

\begin{figure}[H]
	\caption{Basic concept}

	\centering

	\tikzstyle{user} = [draw, ellipse, fill=red!20, node distance=5cm,
		minimum height=2em]
	\tikzstyle{server} = [rectangle, draw, fill=blue!20, text width=5em,
		text centered, rounded corners, minimum height=2em]

	\tikzstyle{request} = [->, thick]
	\tikzstyle{reply} = [->, thick, dashed]
	\tikzstyle{label} = [font=\tiny, text width=2cm]


	\begin{tikzpicture}[node distance = 4cm, auto]
		\node [server] (api) at (0,0) {API};
		\node [server, fill=blue!10] (dns) at (5,2) {DNS};
		\node [server] (proxy) at (5,-2) {Proxy};
		\node [server] (server) at (10,0) {Server};
		\node [user] (user) at (15,0) {User};

		\draw[request] (user.175) to [out=150,in=30] (server.5);
		\node [label, align=center] (cap1) at (12.5, 0.75) {(1) login request};

		\draw[request, color=red!50] (server.155) to [out=130,in=350] (dns.350);
		\node [label, align=right] (cap2) at (8.5, 1.75)
			{(2) DNS request to find proxy};

		\draw[reply, color=red!50] (dns.335) to [out=300,in=170] (server.170);
		\node [label, align=left] (cap3) at (6.5, 0.4) {(3) DNS reply};

		\draw[request] (server.190) to [out=190,in=40] (proxy.25);
		\node [label, align=left] (cap4) at (6.5, -0.6)
			{(4) User credentials request};

		\draw[request] (proxy.155) to [out=130,in=350] (api.350);
		\node [label, align=left, text width=2.3cm] (cap5) at (3.75, -0.2)
			{(5) User credentials request to API};

		\draw[reply] (api.335) to [out=300,in=170] (proxy.170);
		\node [label, align=left, text width=1.4cm] (cap6) at (2, -1.75)
			{(6) User credentials};

		\draw[reply] (proxy.10) to [out=10,in=210] (server.210);
		\node [label, align=left, text width=1.4cm] (cap6) at (8, -1.75)
			{(6) User credentials};

		\draw[reply] (server.355) to [out=330,in=210] (user.185);
		\node [label, align=center] (cap1) at (12.5, -0.75) {(8) login granted};
	\end{tikzpicture}
\end{figure}
