% This file is part of mauth.
%
% mauth is a free document: you can redistribute it and/or modify it under the
% terms of the GNU General Public License as published by the Free Software
% Foundation, either version 3 of the License, or (at your option) any later
% version.
%
% This document is distributed in the hope that it will be useful,but WITHOUT
% ANY WARRANTY; without even the implied warranty of MERCHANTABILITY or
% FITNESS FOR A PARTICULAR PURPOSE. See the GNU General Public License for
% more details.
%
% You should have received a copy of the GNU General Public License along with
% this document. If not, see
%
%   http://www.gnu.org/licenses/
%
%
% Copyright (C)
%   2015-2016 Alexander Haase IT Services <support@alexhaase.de>
%


\section{Idea}

To solve these issues, a centralized database backend should be used, so every
employee is able to manage his own account with low overhead. Usually you'll
already have such a database for your employees credentials to authenticate them
at your companys machines, so why don't use them? These data can be accessed via
a well defined API via HTTPS, so you can feel free to use a backend of your
choice. \\

Usually you'll authenticate yourself with SSH-Keys against the target machine.
Since OpenSSH 6.2\footnote{see
\href{http://www.openssh.com/txt/release-6.2}{release notes for OpenSSH 6.2}},
the \verb+AuthorizedKeysCommand+ is available to support fetching
\verb+authorized_keys+ from a command. With a simple script the keys could be
fetched from the API. \\

Since there is no security benefit in seperating the
maintenance users (since every employee needs sudo on this machine and so would
have access to the other users data), no seperate accounts per user but a
central maintenance account will be used. The real user can be identified via
the used SSH key which should be logged. For privileged access (sudo) onetime
passwords should be used as a second factor of security - integration may be
easily managed via a PAM\footnote{\textbf{P}luggable \textbf{A}uthentication
\textbf{M}odules} module. Thus algorithms like TOTP\footnote{Time-Based One-Time
Password Algorithm is defined in
\href{https://tools.ietf.org/html/rfc6238}{RFC 6238}} are symmetric, not the
client but the credential server should check the onetime password via API. \\

For a faster experience, a caching Proxy may be used, but must be secured via
HTTPS. A simple proxy may be established via nginx as caching reverse proxy for
the API server, secured by TLS.

\subsection{Basic structure}

\begin{figure}[H]
	\caption{Basic concept}

	\centering

	\tikzstyle{user} = [draw, ellipse, fill=red!20, node distance=5cm,
		minimum height=2em]
	\tikzstyle{server} = [rectangle, draw, fill=blue!20, text width=5em,
		text centered, rounded corners, minimum height=2em]

	\tikzstyle{request} = [->, thick]
	\tikzstyle{reply} = [->, thick, dashed]
	\tikzstyle{label} = [font=\tiny, text width=2cm]


	\begin{tikzpicture}[node distance = 4cm, auto]
		\node [server] (api) at (0,0) {API};
		\node [server] (proxy) at (5,-2) {Proxy};
		\node [server] (server) at (10,0) {Server};
		\node [user] (user) at (15,0) {User};

		\draw[request] (user.167) to [out=167,in=15] (server.5);
		\node [label, align=center] (cap1) at (12.75, 0.65) {(1) login request};

		\draw[reply] (server.355) to [out=345,in=193] (user.190);
		\node [label, align=center] (cap1) at (12.75, -0.65)
			{(6) login granted};


		% Way 1: direct access
		\draw[request, color=cyan] (server.155) to [out=155,in=30] (api.30);
		\node [label, align=left] (cap4) at (5, 1.9)
			{(A2) User credentials request};

		\draw[reply, color=cyan] (api.20) to [out=20,in=165] (server.165);
		\node [label, align=left] (cap4) at (5, 0.7)
			{(A3) User credentials};


		% Way 2: via Proxy
		\draw[request] (server.190) to [out=190,in=40] (proxy.25);
		\node [label, align=left] (cap4) at (6.5, -0.6)
			{(2) User credentials request};

		\draw[request] (proxy.155) to [out=130,in=350] (api.350);
		\node [label, align=left, text width=2.3cm] (cap5) at (3.75, -0.2)
			{(3) User credentials request to API};

		\draw[reply] (api.335) to [out=300,in=170] (proxy.170);
		\node [label, align=left, text width=1.4cm] (cap6) at (2, -1.75)
			{(4) User credentials};

		\draw[reply] (proxy.10) to [out=10,in=210] (server.210);
		\node [label, align=left, text width=1.4cm] (cap6) at (8, -1.75)
			{(5) User credentials};


		\draw [decorate, decoration={brace, amplitude=10pt}]
			(9.5,-2.5) -- (0,-2.5) node [black, midway, yshift=-10pt]
			{\tiny HTTPS};
	\end{tikzpicture}
\end{figure}
