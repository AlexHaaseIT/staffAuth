% This file is part of staffAuth.
%
% staffAuth is a free document: you can redistribute it and/or modify it under
% the terms of the GNU General Public License as published by the Free Software
% Foundation, either version 3 of the License, or (at your option) any later
% version.
%
% This document is distributed in the hope that it will be useful,but WITHOUT
% ANY WARRANTY; without even the implied warranty of MERCHANTABILITY or
% FITNESS FOR A PARTICULAR PURPOSE. See the GNU General Public License for
% more details.
%
% You should have received a copy of the GNU General Public License along with
% this document. If not, see
%
%   http://www.gnu.org/licenses/
%
%
% Copyright (C)
%   2015-2016 Alexander Haase IT Services <support@alexhaase.de>
%


\section{Idea}

To solve these issues, a centralized database backend should be used, so every
employee is able to manage his own account with low overhead. Usually you'll
already have a central storage for your employees credentials to authenticate
them at your companys machines, so why don't use them? These data should be
accessed via a well defined API from outside, so you're free to use a backend of
your choice. \\

In general no passwords should be used at you customers' side. For login into
machines SSH-keys should be used, thus this is an asymmetric system were the
public key may be public accessible (which in case will hapen via API). For
privileged access (sudo) onetime passwords should be used. Thus algorithms like
TOTP are symmetric, not the client but the credential server should check the
onetime password via API.
