% This file is part of staffAuth.
%
% staffAuth is a free document: you can redistribute it and/or modify it under
% the terms of the GNU General Public License as published by the Free Software
% Foundation, either version 3 of the License, or (at your option) any later
% version.
%
% This document is distributed in the hope that it will be useful,but WITHOUT
% ANY WARRANTY; without even the implied warranty of MERCHANTABILITY or
% FITNESS FOR A PARTICULAR PURPOSE. See the GNU General Public License for
% more details.
%
% You should have received a copy of the GNU General Public License along with
% this document. If not, see
%
%   http://www.gnu.org/licenses/
%
%
% Copyright (C)
%   2015-2016 Alexander Haase IT Services <support@alexhaase.de>
%


\colorlet{punct}{red!60!black}
\definecolor{delim}{RGB}{20,105,176}
\colorlet{numb}{magenta!60!black}

\lstdefinelanguage{http_request}{
	basicstyle=\small\ttfamily,
	showstringspaces=false,
	breaklines=true,
	backgroundcolor=\color{box_bg},
	literate=
		{GET}{{{\color{delim}{GET}}}}{3}
		{:login}{{{\color{numb}{:login}}}}{5}
}

\lstdefinelanguage{json}{
	basicstyle=\small\ttfamily,
	showstringspaces=false,
	breaklines=true,
	backgroundcolor=\color{box_bg},
	literate=
		*{0}{{{\color{numb}0}}}{1}
		{1}{{{\color{numb}1}}}{1}
		{2}{{{\color{numb}2}}}{1}
		{3}{{{\color{numb}3}}}{1}
		{4}{{{\color{numb}4}}}{1}
		{5}{{{\color{numb}5}}}{1}
		{6}{{{\color{numb}6}}}{1}
		{7}{{{\color{numb}7}}}{1}
		{8}{{{\color{numb}8}}}{1}
		{9}{{{\color{numb}9}}}{1}
		{:}{{{\color{punct}{:}}}}{1}
		{,}{{{\color{punct}{,}}}}{1}
		{\{}{{{\color{delim}{\{}}}}{1}
		{\}}{{{\color{delim}{\}}}}}{1}
		{[}{{{\color{delim}{[}}}}{1}
		{]}{{{\color{delim}{]}}}}{1}
}




\section{API}

To access the employees credentials and other data, a well defined API must be
used.

\subsection{HTTPS}

To ensure data authenticity, all HTTP traffic must be secured with an up-to-date
version of TLS (HTTPS). As for any other HTTPS server, the servers' certificate
must be signed by a CA\footnote{\textbf{C}ertificate \textbf{A}uthority} known
by the client servers and be checked against this CA.


\subsection{Schema}

All data is sent and received as JSON. \\

For GET requests, all parameters are embeded as a segment in the endpoint. An
example API curl request for the path \verb+users/:username+ with parameter
\verb+username+ set to \underline{max} and API server \verb+api.example.com+
may look like this:

\begin{lstlisting}[language=none, numbers=none]
:~$ curl https://api.example.com/keys
[
  {
    "id": 1,
    "key": "ssh-rsa AAA..."
...
\end{lstlisting}

For POST requests, parameters not included in the path should be encoded as JSON
with a Content-Type of \verb+application/json+. An example API curl request for
the path \verb+users/:username/verify/otp+ with parameter \verb+username+ and
\verb+token+ may look like this:

\begin{lstlisting}[language=none, numbers=none]
:~$ curl -d '{"token":123456}' \
        https://api.example.com/users/max/verify/otp
{
  "verified": true
}
\end{lstlisting}


\subsection{Endpoints}

\newcommand{\includeApiDefinition}[1]{
	\begin{minipage}{\textwidth}
	\input{api/#1}
	\end{minipage}
}


\includeApiDefinition{ssh_keys}
\includeApiDefinition{ssh_keys_notify}
\includeApiDefinition{verify_otp}


\subsection{Technical notes}

The API server should be secured against brute-force and denial-of-service
attacks. The following points are recommended:

\begin{itemize}
	\item Rate-limit requests
	\item Block unauthorized IPs and keep blacklists up-to-date.
	\item Only grant access to authorized clients e.g. by client TLS
		certificates or API tokens.
\end{itemize}
