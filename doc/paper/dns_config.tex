% This file is part of staffAuth.
%
% staffAuth is a free document: you can redistribute it and/or modify it under
% the terms of the GNU General Public License as published by the Free Software
% Foundation, either version 3 of the License, or (at your option) any later
% version.
%
% This document is distributed in the hope that it will be useful,but WITHOUT
% ANY WARRANTY; without even the implied warranty of MERCHANTABILITY or
% FITNESS FOR A PARTICULAR PURPOSE. See the GNU General Public License for
% more details.
%
% You should have received a copy of the GNU General Public License along with
% this document. If not, see
%
%   http://www.gnu.org/licenses/
%
%
% Copyright (C)
%   2015-2016 Alexander Haase IT Services <support@alexhaase.de>
%


\section{Configuration via DNS}

\textit{Read this section \underline{carefuly} before implementing this in your
network. It may cause a huge vulnerability if used in wrong environments!} \\

To decrease maintenance overhead, the API proxy may be found via predefined DNS
records. To find the local API proxy, an NAPTR\footnote{NAPTR resource record is
defined in \href{https://tools.ietf.org/html/rfc2915}{RFC 2915}} resource record
named \verb+staffauth+ placed in the search-domain of the client may be used to
deploy the domain name of the API proxy: \\

\begin{lstlisting}[language=none, numbers=none]
staffauth.example.com.  IN  NAPTR  100  10  "A"  ""  ""  saprox.example.com.
\end{lstlisting}

~\\

Unlike HTTPS plain DNS \underline{has no builtin authenticity checks}.
Use this configuration method only, if you can ensure that no third party is
able to manipulate your network or if your DNS records are secured by
DNSSEC and your resolver checks these records carefuly.
