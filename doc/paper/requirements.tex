% This file is part of staffAuth.
%
% staffAuth is a free document: you can redistribute it and/or modify it under
% the terms of the GNU General Public License as published by the Free Software
% Foundation, either version 3 of the License, or (at your option) any later
% version.
%
% This document is distributed in the hope that it will be useful,but WITHOUT
% ANY WARRANTY; without even the implied warranty of MERCHANTABILITY or
% FITNESS FOR A PARTICULAR PURPOSE. See the GNU General Public License for
% more details.
%
% You should have received a copy of the GNU General Public License along with
% this document. If not, see
%
%   http://www.gnu.org/licenses/
%
%
% Copyright (C)
%   2015-2016 Alexander Haase IT Services <support@alexhaase.de>
%


\section{Requirements}

To solve these issues, a centralized database backend should be used, so every
employee is able to manage his own account with low overhead. \\

The following requirements should be met:

\begin{itemize}
	\item Data initegrity and authenticity must be ensured. Otherwise one could
		simply give himself sudo rights. Even third party could get access via
		MITM\footnote{\textbf{M}an \textbf{I}n \textbf{T}he \textbf{M}iddle}
		attacks, if the data could be manipulated.
	\item Not every server should be configured by hand. This is time consuming
		and annoying. The system should find its configuration automaticaly, but
		it must be authenticated.
	\item Only one account should be used per employee.
	\item Several authentication servers should be able to be joined. In some
		special cases you have e.g. one company which maintains the network and
		another for some of the servers. The last one may need e.g. an SSH
		tunnel or access to nagios, etc.
\end{itemize}
