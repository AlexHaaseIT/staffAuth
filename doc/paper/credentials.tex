% This file is part of staffAuth.
%
% staffAuth is a free document: you can redistribute it and/or modify it under
% the terms of the GNU General Public License as published by the Free Software
% Foundation, either version 3 of the License, or (at your option) any later
% version.
%
% This document is distributed in the hope that it will be useful,but WITHOUT
% ANY WARRANTY; without even the implied warranty of MERCHANTABILITY or
% FITNESS FOR A PARTICULAR PURPOSE. See the GNU General Public License for
% more details.
%
% You should have received a copy of the GNU General Public License along with
% this document. If not, see
%
%   http://www.gnu.org/licenses/
%
%
% Copyright (C)
%   2015-2016 Alexander Haase IT Services <support@alexhaase.de>
%


\section{Accessing user credentials}

To make the user credentials of all employees available on all customer
machines, the NSS\footnote{\textbf{N}ame \textbf{S}ervice \textbf{S}witch}
plugin system could be used. A new plugin could query all credentials via the
proxy server.


\subsection{User IDs}

Linux has $2^{32}$ available UIDs\footnote{\textbf{U}ser \textbf{ID}entifier},
so we can add all maintainer user accounts without getting in trouble with the
UIDs at customer site. Normal Linux UIDs are in the range 1000-65535, while
subUIDs start at 100000. The resulting slot (65536-99999) could be used for the
employees accounts. The UID will be shifted to this slot on-the-fly by the NSS
plugin.


\subsection{Group memberships}

At the moment there is no good solution to grant membership to local groups to
remote users with NSS only. But instead you can use pam\_group
PAM\footnote{\textbf{P}luggable \textbf{A}uthentication \textbf{M}odules} module
to grant access to system groups for your needs. Some groups like
\verb+systemd-journal+ may be set for every user, others depending on other
groups of the user which will be retrieved by NSS.
