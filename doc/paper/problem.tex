% This file is part of mauth.
%
% mauth is a free document: you can redistribute it and/or modify it under the
% terms of the GNU General Public License as published by the Free Software
% Foundation, either version 3 of the License, or (at your option) any later
% version.
%
% This document is distributed in the hope that it will be useful,but WITHOUT
% ANY WARRANTY; without even the implied warranty of MERCHANTABILITY or
% FITNESS FOR A PARTICULAR PURPOSE. See the GNU General Public License for
% more details.
%
% You should have received a copy of the GNU General Public License along with
% this document. If not, see
%
%   http://www.gnu.org/licenses/
%
%
% Copyright (C)
%   2015-2016 Alexander Haase IT Services <support@alexhaase.de>
%


\section{Problem}

When maintaining several machines on the customers side, you can choose between
some options to restrict the privileged access (like sudo) to authorized staff
only, but they have some issues:

\begin{enumerate}
	\item There is a separate maintenance account on the target machine with a
		fixed password which is not shared between customers or machines.
		\begin{itemize}
			\item If your customer gets the password, he has full access on the
				target machine until you change the password. After you have
				changed it, you have to inform all your employees about the new
				password and if you have used the same password all over the
				customers' network, this is very time consuming.
			\item If an employee left your company, you have to change all
				passwords he could know. This is very time consuming, so you'll
				never want to do that until you really have to, so mostly he'll
				have access to this machine until the rest of the lifetime of
				it.
			\item You have to maintain a password-database for all the
				credentials and access privileges for your employees, which is a
				huge overhead.
		\end{itemize}

	\item There is a separate maintenance account on the target machine with a
		\ul{fixed password for all machines accross all customers}. In addition
		to issue 1 you gain access to \ul{all} machines accross all of your
		customers if you get the password. Regardless to say that this is a huge
		hole in your security concept, but some companies handle it this way.

	\item Every employee has its own account on the target machine. The
		password should be the same over all machines, so the employee won't
		forget all these passwords.
		\begin{itemize}
			\item If you get a new employee or one left, you have to add /
				delete the accounts on all machines for which the employee needs
				access. This is \textit{really} time consuming. In most cases
				this will end up in a single account per machine, which has
				security issues described above.
			\item It is very time consuming to change the password for a single
				user. As more machines you have to maintain, as more time you'll
				need to change your password on any machine.
		\end{itemize}
\end{enumerate}

No matter which solution you use, it gets really complicated at least with an
increasing number of machines you are maintaining.
