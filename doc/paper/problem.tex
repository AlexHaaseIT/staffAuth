% This file is part of staffAuth.
%
% staffAuth is a free document: you can redistribute it and/or modify it under
% the terms of the GNU General Public License as published by the Free Software
% Foundation, either version 3 of the License, or (at your option) any later
% version.
%
% This document is distributed in the hope that it will be useful,but WITHOUT
% ANY WARRANTY; without even the implied warranty of MERCHANTABILITY or
% FITNESS FOR A PARTICULAR PURPOSE. See the GNU General Public License for
% more details.
%
% You should have received a copy of the GNU General Public License along with
% this document. If not, see
%
%   http://www.gnu.org/licenses/
%
%
% Copyright (C)
%   2015 Alexander Haase IT Services <support@alexhaase.de>
%


\section{Problem}

When maintaining several servers and/or clients, you can choose between some
options to restrict the access for maintenance functionality (like sudo in
Linux / Unix) to authorized staff only, but they have some issues:

\begin{enumerate}
	\item There is a seperate maintenance account on the target maching with a
		fixed password.
		\begin{itemize}
			\item If your customer gets the password, he has full access on the
				target machine until you change the password. After you have
				changed it, you have to inform all your employees about the new
				password and if you have used the same password all over the
				network, this is very time consuming.
			\item If an employee left your company, you have to change all
				passwords he could know. This is very time consuming, so you'll
				never do that until you really have to, so mostly he'll have
				access to this machine until the livetime of it.
		\end{itemize}

	\item Every employee has its own account on the target machine. The
		password should be the same over all machines, so the employee won't
		forget all these passwords.
		\begin{itemize}
			\item If you get a new employee or one left, you have to add /
				delete the accounts on all machines where the employee needs
				access. This is \textit{really} time consuming.
			\item It is very time consuming to change the password for a single
				user. As more machines you have to maintain, as more time you'll
				need.
			\item If you use a solution like LDAP or any other database for
				the users credentials, every machine needs a connection to your
				database with its own credentials and you have to open your
				database connection for the world wide web, which might be a
				security issue.
		\end{itemize}
\end{enumerate}

No matter which solution you use, it gets really complicated at least with an
increasing number of machines you are maintaining.
